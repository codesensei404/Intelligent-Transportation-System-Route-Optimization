\documentclass[12pt]{article}
\usepackage{amsmath, amssymb}
\usepackage{enumitem}

\begin{document}

\section*{Lecture 6 Scribe – Probability \& Random Variables}
\textit{(Exam Preparation Notes)}

\hrule
\vspace{0.5cm}

\section*{1. Recap: Random Variables and Independent Events}

\subsection*{Random Variables}

A random variable (RV) is a numerical description of the outcome of a random experiment.

\begin{itemize}
\item Instead of studying outcomes directly, we assign numbers to them.
\item A discrete random variable takes countable values.
\end{itemize}

If $(X)$ is a random variable, then for each value $(x)$:
\[
P(X=x)
\]
represents the probability that the random variable takes that value.

This probability assignment must satisfy:

\begin{enumerate}
\item $(P(X=x) \ge 0)$
\item $(\sum_x P(X=x) = 1)$
\end{enumerate}

These conditions ensure that the probabilities are valid and account for all possible outcomes.

\hrule
\vspace{0.5cm}

\subsection*{Independent Events}

Two events $(A)$ and $(B)$ are independent if:
\[
P(A \cap B) = P(A)P(B)
\]

\subsubsection*{Reasoning}

\begin{itemize}
\item If knowing that $(A)$ occurred does not change the probability of $(B)$, then the events are independent.
\item The multiplication rule follows from the idea that both events occur without influencing each other.
\end{itemize}

For random variables, independence means:
\[
P(X=x, Y=y) = P(X=x),P(Y=y)
\]

\section*{2. Discrete Random Variable Types}

\subsection*{(A) Bernoulli Random Variable}

\subsubsection*{Definition}

A Bernoulli RV represents one experiment with only two outcomes:

\begin{itemize}
\item Success (1)
\item Failure (0)
\end{itemize}

Let:

\begin{itemize}
\item $(P(\text{success}) = p)$
\item $(P(\text{failure}) = 1-p)$
\end{itemize}

\subsubsection*{Probability Distribution}

\[
P(X=1)=p,\quad P(X=0)=1-p
\]

\subsubsection*{Logical Basis}

Since there are only two outcomes:
\[
p + (1-p) = 1
\]

This satisfies the total probability rule.

\subsection*{(B) Binomial Random Variable}

\subsubsection*{Definition}

Counts the number of successes in $(n)$ independent Bernoulli trials.

Each trial:

\begin{itemize}
\item Same success probability $(p)$
\item Independent of others
\end{itemize}

\subsubsection*{Derivation of the Formula}

We want the probability of exactly $(k)$ successes.

Step 1: Select which $(k)$ trials are successes

Number of ways:
\[
\binom{n}{k}
\]

Step 2: Probability of one specific arrangement

\begin{itemize}
\item $(k)$ successes $\rightarrow$ $(p^k)$
\item $(n-k)$ failures $\rightarrow$ $((1-p)^{n-k})$
\end{itemize}

Step 3: Multiply by number of arrangements
\[
P(X=k)=\binom{n}{k} p^k (1-p)^{n-k}
\]

This accounts for all ways to obtain $(k)$ successes.

\subsection*{(C) Geometric Random Variable}

\subsubsection*{Definition}

Represents the number of trials required until the first success occurs.

\subsubsection*{Derivation}

For the first success to occur on trial $(k)$:

\begin{itemize}
\item Trials 1 to $(k-1)$: all failures\\
Probability = $((1-p)^{k-1})$

\item Trial $(k)$: success\\
Probability = $(p)$
\end{itemize}

Multiply:
\[
P(X=k)=(1-p)^{k-1}p
\]

\subsubsection*{Interpretation}

The probability decreases as $(k)$ increases because more failures must occur before the first success.

\subsection*{(D) Poisson Random Variable}

\subsubsection*{Definition}

Models the number of occurrences of an event in a fixed interval when:

\begin{itemize}
\item Events occur independently
\item The average rate is constant
\end{itemize}

Let average number of events = $(\lambda)$

\subsubsection*{Probability Formula}

\[
P(X=k)=\frac{e^{-\lambda}\lambda^k}{k!}
\]

\subsubsection*{Derivation Idea}

Poisson arises as a limit of the binomial distribution when:

\begin{itemize}
\item $(n)$ is large
\item $(p)$ is small
\item $(np = \lambda)$ remains constant
\end{itemize}

As $(n \to \infty)$, the binomial expression simplifies to the Poisson form.

\section*{3. Expectation}

\subsection*{Definition}

The expectation (mean) of a discrete random variable is:
\[
E[X]=\sum x,P(X=x)
\]

\subsubsection*{Reasoning}

Each value is weighted by how likely it is to occur.

This gives the long-run average value.

\subsection*{Expectation of a Function}

If $(Y=g(X))$:
\[
E[g(X)] = \sum g(x),P(X=x)
\]

\subsubsection*{Logic}

\begin{enumerate}
\item Transform values using $(g(x))$
\item Then take the weighted average.
\end{enumerate}

\subsection*{Linear Operations}

\subsubsection*{Property 1}

\[
E[aX+b]=aE[X]+b
\]

\subsubsection*{Property 2}

\[
E[X+Y]=E[X]+E[Y]
\]

\subsubsection*{Reasoning}

The expectation operator distributes across sums and constants because summation is linear.

\section*{4. Moments}

Moments describe the shape and spread of a distribution.

\subsection*{nth Moment (About Origin)}

\[
E[X^n]
\]

This measures the average of the nth power of the random variable.

\subsection*{Central Moments}

Central moments measure variation around the mean.

Let:
\[
\mu = E[X]
\]

The nth central moment:
\[
E[(X-\mu)^n]
\]

\subsubsection*{Variance (2nd Central Moment)}

\[
\text{Var}(X)=E[(X-\mu)^2]
\]

Alternative form:
\[
\text{Var}(X)=E[X^2]-(E[X])^2
\]

\subsubsection*{Meaning}

Variance measures how spread out the values are from the mean.

\subsubsection*{Skewness (3rd Central Moment)}

Describes asymmetry:

\begin{itemize}
\item Positive skew $\rightarrow$ long right tail
\item Negative skew $\rightarrow$ long left tail
\end{itemize}

\subsubsection*{Kurtosis (4th Central Moment)}

Measures peakedness:

\begin{itemize}
\item High kurtosis $\rightarrow$ sharp peak
\item Low kurtosis $\rightarrow$ flatter distribution
\end{itemize}

\section*{5. CDF and PDF}

\subsection*{CDF (Cumulative Distribution Function)}

Definition:
\[
F(x)=P(X \le x)
\]

\subsubsection*{Properties}

\begin{enumerate}
\item Non-decreasing function
\item $(0 \le F(x) \le 1)$
\item Approaches 1 as $(x)$ becomes large
\end{enumerate}

\subsubsection*{Interpretation}

Gives total probability accumulated up to value $(x)$.

\subsection*{PMF (Discrete Case)}

For a discrete RV:
\[
P(X=x)
\]

Relation to CDF:
\[
F(x)=\sum_{t \le x} P(X=t)
\]

So the CDF is the cumulative sum of probabilities.

\section*{6. Applications of Poisson Distribution}

\subsection*{Example 1: Misprints in a Book}

If:

\begin{itemize}
\item Average number of misprints per page = $(\lambda)$
\end{itemize}

Then the probability of exactly $(k)$ misprints on a page is:
\[
P(X=k)=\frac{e^{-\lambda}\lambda^k}{k!}
\]

\subsubsection*{Why Poisson fits}

\begin{itemize}
\item Misprints occur randomly
\item Independent occurrences
\item Constant average rate
\end{itemize}

\subsection*{Example 2: Phone Calls / Phone Numbers}

If:

\begin{itemize}
\item Calls arrive randomly
\item Average rate is known
\end{itemize}

Then the number of calls in a fixed time interval follows a Poisson distribution.

\subsubsection*{Reasoning}

\begin{itemize}
\item Calls occur independently
\item Rate remains constant
\item Count of events in time interval $\rightarrow$ Poisson model
\end{itemize}

\end{document}
