\documentclass[12pt]{article}
\usepackage{amsmath, amssymb}

\begin{document}

\section*{Lecture 6 (L6) – Probability Theory}
\section*{Exam Preparation Scribe}

\section*{1. Previous Lecture Recap: Random Variables (RVs)}

\textbf{Definition: Random Variable}

A random variable (RV) is a real-valued function defined on the sample space of a random experiment.

It assigns a numerical value to each outcome.

Usually denoted by capital letters such as $X, Y$.

\textbf{Types of Random Variables}

Discrete Random Variable: Takes countable values.

Continuous Random Variable: Takes values over a continuous interval.

\section*{2. Independent Events}

\textbf{Definition}

Two events $A$ and $B$ are said to be independent if

\[
P(A \cap B) = P(A)P(B)
\]

This means that the occurrence of one event does not affect the probability of the other.

\textbf{Example}

If a coin is tossed and a die is rolled:

Event $A$: Coin shows Head

Event $B$: Die shows 3

Since the coin toss does not affect the die roll,

\[
P(A \cap B) = P(A)P(B)
\]

\section*{3. Types of Discrete Random Variables}

\subsection*{3.1 Bernoulli Random Variable}

\textbf{Definition}

A Bernoulli random variable represents a single experiment with two outcomes:

Success (1) with probability $p$

Failure (0) with probability $1-p$

\textbf{Probability Mass Function (PMF)}

\[
P(X=x)=
\begin{cases}
p, & x=1 \\
1-p, & x=0
\end{cases}
\]

\subsection*{3.2 Binomial Random Variable}

\textbf{Definition}

A Binomial random variable counts the number of successes in $n$ independent Bernoulli trials with success probability $p$.

\textbf{PMF}

\[
P(X=k)=\binom{n}{k}p^{k}(1-p)^{n-k}, \quad k=0,1,\ldots,n
\]

\subsection*{3.3 Geometric Random Variable}

\textbf{Definition}

A Geometric random variable represents the number of trials needed to get the first success.

\textbf{PMF}

\[
P(X=k)=(1-p)^{k-1}p, \quad k=1,2,3,\ldots
\]

\subsection*{3.4 Poisson Random Variable}

\textbf{Definition}

A Poisson random variable models the number of occurrences of an event in a fixed interval.

\textbf{PMF}

\[
P(X=k)=\frac{\lambda^{k}e^{-\lambda}}{k!}, \quad k=0,1,2,\ldots
\]

where $\lambda>0$ is the mean rate.

\section*{4. Expectation of Random Variables}

\subsection*{4.1 Definition of Expectation}

For a discrete random variable $X$,

\[
E[X]=\sum_{x} xP(X=x)
\]

\textbf{Example}

If

\[
P(X=0)=\frac{1}{2}, \quad P(X=1)=\frac{1}{2}
\]

then

\[
E[X]=0\cdot\frac{1}{2}+1\cdot\frac{1}{2}=\frac{1}{2}
\]

\subsection*{4.2 Expectation of a Function of an RV}

If $Y=g(X)$, then

\[
E[g(X)]=\sum_{x} g(x)P(X=x)
\]

\subsection*{4.3 Linear Operation with Expectation}

For constants $a,b$,

\[
E[aX+b]=aE[X]+b
\]

\section*{5. Moments of Random Variables}

\subsection*{5.1 nth Moment}

The nth moment of $X$ is

\[
E[X^{n}]
\]

\subsection*{5.2 Central Moments}

\textbf{Variance}

\[
\text{Var}(X)=E[(X-E[X])^{2}]=E[X^{2}]-(E[X])^{2}
\]

\textbf{Skewness}

The third central moment:

\[
E[(X-E[X])^{3}]
\]

\textbf{Kurtosis}

The fourth central moment:

\[
E[(X-E[X])^{4}]
\]

\section*{6. Cumulative Distribution Function (CDF)}

\textbf{Definition}

The CDF of a random variable $X$ is

\[
F_X(x)=P(X\le x)
\]

\textbf{Properties}

\[
0 \le F_X(x) \le 1
\]

$F_X(x)$ is non-decreasing

\[
\lim_{x\to -\infty} F_X(x)=0
\]

\[
\lim_{x\to \infty} F_X(x)=1
\]

\textbf{Example}

For a discrete RV,

\[
F_X(x)=\sum_{t\le x} P(X=t)
\]

\section*{7. Probability Density Function (PDF)}

\textbf{Definition}

For a continuous random variable $X$, the PDF $f_X(x)$ satisfies

\[
P(a\le X\le b)=\int_{a}^{b} f_X(x)\,dx
\]

\textbf{Properties}

\[
f_X(x)\ge 0
\]

\[
\int_{-\infty}^{\infty} f_X(x)\,dx=1
\]

\[
F_X(x)=\int_{-\infty}^{x} f_X(t)\,dt
\]

\textbf{Example}

If

\[
f_X(x)=
\begin{cases}
1, & 0\le x\le 1 \\
0, & \text{otherwise}
\end{cases}
\]

then

\[
F_X(x)=
\begin{cases}
0, & x<0 \\
x, & 0\le x\le 1 \\
1, & x>1
\end{cases}
\]

\end{document}