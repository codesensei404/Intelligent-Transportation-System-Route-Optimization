\documentclass[12pt]{article}
\usepackage{amsmath, amssymb}
\usepackage{geometry}
\geometry{a4paper, margin=1in}

\begin{document}

\begin{center}
    \LARGE \textbf{Lecture Scribe: Randomized Min-Cut Algorithm}
\end{center}

\noindent
\textbf{Course:} CSE400 – Fundamentals of Probability in Computing \\
\textbf{Lecture:} 10 \\
\textbf{Instructor:} Dhaval Patel, PhD \\
\textit{(Prepared strictly from the provided PDF)}

\section*{1. Min-Cut Problem}

\subsection*{1.1 Why Use Min-Cut?}

Min-cut algorithms are used to solve problems related to:
\begin{itemize}
    \item \textbf{Network connectivity}
    \item \textbf{Network reliability}
    \item \textbf{Network optimization}
\end{itemize}

The goal is to identify \textbf{critical edges} whose removal disconnects the network with \textbf{minimum cost or size}.

\subsubsection*{Applications of Min-Cut}

\begin{enumerate}
    \item \textbf{Network Design}
    \begin{itemize}
        \item Used to improve communication efficiency.
        \item Helps find the \textbf{minimum capacity cut} in a network.
    \end{itemize}

    \item \textbf{Communication Networks}
    \begin{itemize}
        \item Helps analyze \textbf{network vulnerability to failures}.
        \item Supports the design of \textbf{fault-tolerant and robust networks}.
    \end{itemize}

    \item \textbf{VLSI Design}
    \begin{itemize}
        \item Used for \textbf{partitioning circuits} into smaller components.
        \item Reduces \textbf{interconnectivity complexity}.
    \end{itemize}
\end{enumerate}

\section*{2. What is Min-Cut?}

\subsection*{2.1 Definition: Cut-Set}

A \textbf{cut-set} in a graph is:

\begin{quote}
A set of edges whose removal breaks the graph into two or more connected components.
\end{quote}

\subsection*{2.2 Definition: Minimum Cut}

Given a graph
\[
G = (V, E)
\]
with $n$ vertices,
\begin{itemize}
    \item The \textbf{minimum cut (min-cut)} problem is to find a \textbf{cut-set with minimum cardinality}.
\end{itemize}

\subsection*{2.3 Important Observation}

\begin{itemize}
    \item Min-cut algorithms such as \textbf{Karger’s algorithm} are \textbf{randomized}.
    \item Their result depends on \textbf{which edges are selected early}.
    \item If \textbf{critical edges} are contracted early, the algorithm may \textbf{fail to find the true minimum cut}.
\end{itemize}

\section*{3. Edge Contraction}

\subsection*{3.1 Definition: Edge Contraction}

The \textbf{main operation} used in min-cut algorithms is \textbf{edge contraction}.

\subsection*{3.2 Step-by-Step Edge Contraction Process}

For an edge $(u, v)$:
\begin{enumerate}
    \item Vertices $u$ and $v$ are merged into a \textbf{single vertex}.
    \item All edges between $u$ and $v$ are removed.
    \item All remaining edges are retained.
    \item The resulting graph:
    \begin{itemize}
        \item May contain \textbf{parallel edges}
        \item Contains \textbf{no self-loops}
    \end{itemize}
\end{enumerate}

\section*{4. Min-Cut Runs}

\subsection*{4.1 Successful Min-Cut Run}

A \textbf{successful min-cut run} refers to:
\begin{itemize}
    \item An execution of the algorithm that \textbf{correctly identifies the minimum cut} of the graph.
\end{itemize}

(Shown graphically in the lecture slides.)

\subsection*{4.2 Unsuccessful Min-Cut Run}

An \textbf{unsuccessful min-cut run} refers to:
\begin{itemize}
    \item An execution where the algorithm \textbf{fails to identify the minimum cut}.
    \item This typically happens when \textbf{important edges are contracted too early}.
\end{itemize}

(Shown graphically in the lecture slides.)

\section*{5. Max-Flow Min-Cut Theorem}

\subsection*{5.1 Theorem Statement}

\begin{quote}
In a flow network, the maximum amount of flow passing from the source to the sink is equal to the total weight of the edges in a minimum cut.
\end{quote}

\subsection*{5.2 Definitions Used in the Theorem}

\begin{enumerate}
    \item \textbf{Capacity of a Cut}
    \begin{itemize}
        \item The sum of capacities of edges oriented from:
        \begin{itemize}
            \item vertex $\in X$ to vertex $\in Y$
        \end{itemize}
    \end{itemize}

    \item \textbf{Minimum Cut}
    \begin{itemize}
        \item The cut with the \textbf{smallest capacity}
    \end{itemize}

    \item \textbf{Minimum Cut Capacity}
    \begin{itemize}
        \item The capacity value of the minimum cut
    \end{itemize}

    \item \textbf{Maximum Flow}
    \begin{itemize}
        \item The largest possible flow from \textbf{source $S$} to \textbf{sink $T$}
    \end{itemize}
\end{enumerate}

\section*{6. Deterministic Min-Cut Algorithm}

\subsection*{6.1 Stoer–Wagner Min-Cut Algorithm}

\subsubsection*{Theorem Used}

Let $s$ and $t$ be two vertices of graph $G$.

Let
\[
G/\{s,t\}
\]
be the graph obtained by \textbf{merging $s$ and $t$}.

A minimum cut of $G$ is obtained by taking the \textbf{smaller} of:
\begin{enumerate}
    \item A \textbf{minimum $s$–$t$ cut} of $G$
    \item A \textbf{minimum cut of} $G/\{s,t\}$
\end{enumerate}

\subsubsection*{Logical Reasoning (As Given in Lecture)}

There are two cases:
\begin{enumerate}
    \item \textbf{Case 1:} \\
    A minimum cut of $G$ separates $s$ and $t$ \\
    $\rightarrow$ Then the \textbf{minimum $s$–$t$ cut} is the minimum cut of $G$

    \item \textbf{Case 2:} \\
    No minimum cut separates $s$ and $t$ \\
    $\rightarrow$ Then the \textbf{minimum cut of} $G/\{s,t\}$ is valid
\end{enumerate}

Hence, the theorem holds.

\subsection*{6.2 Pseudocode}

\subsubsection*{Algorithm 1: MinimumCutPhase$(G, a)$}

\textbf{Step-by-Step}
\begin{enumerate}
    \item Initialize:
    \[
    A \leftarrow \{a\}
    \]
    \item While $A \neq V$:
    \begin{itemize}
        \item Add to $A$ the \textbf{most tightly connected vertex}
    \end{itemize}
    \item Return:
    \begin{itemize}
        \item The \textbf{cut weight}, called the \textit{cut of the phase}
    \end{itemize}
\end{enumerate}

\subsubsection*{Algorithm 2: MinimumCut$(G)$}

\textbf{Step-by-Step}
\begin{enumerate}
    \item While $|V| \ge 1$:
    \begin{itemize}
        \item Choose any vertex $a \in V$
        \item Execute \texttt{MinimumCutPhase(G, a)}
    \end{itemize}
    \item If the cut-of-the-phase is lighter than the current minimum cut:
    \begin{itemize}
        \item Store it as the current minimum cut
    \end{itemize}
    \item Shrink the graph by:
    \begin{itemize}
        \item Merging the \textbf{last two vertices added}
    \end{itemize}
    \item Return the \textbf{minimum cut}
\end{enumerate}

\section*{7. Randomized Min-Cut Algorithm}

\subsection*{7.1 Why Randomized Algorithms?}

Randomized algorithms provide:
\begin{itemize}
    \item \textbf{Probabilistic guarantees of success}
    \item \textbf{Accurate estimates} with fewer iterations
\end{itemize}

\subsubsection*{Advantages Mentioned}

\begin{itemize}
    \item Efficiency
    \item Parallelization
    \item Approximation guarantees
    \item Avoidance of worst-case instances
    \item Heuristic nature
    \item Robustness
\end{itemize}

\subsection*{7.2 Karger’s Randomized Algorithm}

\begin{itemize}
    \item A \textbf{randomized algorithm} for finding the minimum cut
    \item Uses \textbf{random edge contraction}
    \item Success probability increases with repeated runs
\end{itemize}

(Pseudocode shown in lecture slides.)

\section*{8. Deterministic vs Randomized Min-Cut}

\subsection*{Deterministic Min-Cut}

\begin{itemize}
    \item Always guarantees \textbf{exact minimum cut}
    \item Higher time complexity for large graphs
    \item \textbf{Stoer–Wagner complexity}:
    \[
    O(V \cdot E + V^2 \log V)
    \]
\end{itemize}

\subsection*{Randomized Min-Cut}

\begin{itemize}
    \item Produces \textbf{approximate minimum cut with high probability}
    \item \textbf{Karger’s algorithm complexity}:
    \[
    O(V^2)
    \]
\end{itemize}

\section*{9. Theorem for Randomized Min-Cut}

\subsection*{Theorem Statement}

The randomized min-cut algorithm outputs a \textbf{minimum cut set} with probability at least:
\[
\frac{2}{n(n-1)}
\]
where $n$ is the number of vertices.

\section*{10. Python Simulation (Lecture Activity)}

\begin{itemize}
    \item Students are instructed to:
    \begin{itemize}
        \item Open the \textbf{Campuswire post for Lecture 10}
        \item Download the provided \textbf{\texttt{.ipynb} file}
        \item Run the simulation to observe randomized min-cut behavior
    \end{itemize}
\end{itemize}

\section*{End of Lecture}

\begin{center}
\textbf{Thank You}
\end{center}

\end{document}
